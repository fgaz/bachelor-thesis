\chapter{Intro}
\label{cha:intro}

\section{Context}
\label{sec:context}

\section{Problem}
\label{sec:problem}

\chapter{Solution}
\label{sec:solution}

\cite{refinement-types-for-haskell}

\chapter{Results}
\label{cha:results}

\begin{figure}
  \begin{minipage}{0.5\textwidth}
    \begin{tikzpicture}
    \begin{axis}
      [ title = Unoptimized list reverse
      , xlabel = List length
      , ylabel = Time (s)
      , error bars/y dir = both
      , error bars/y explicit
      , error bars/error bar style = { line width=0.5pt }
      , error bars/error mark options =
          { rotate=90
          , red
          , mark size=4pt
          , line width=0.5pt
          }
      ]
      \addplot table
        [ x = N
        , y = Mean
        , y error = Stddev
        , col sep = comma
        ]
        {benchmark-unoptimized.csv};
    \end{axis}
    \end{tikzpicture}
    \caption{A subfigure}\label{fig:bench1:a}
  \end{minipage}
  \begin{minipage}{0.5\textwidth}
    \begin{tikzpicture}
    \begin{axis}
      [ title = Optimized list reverse
      , xlabel = List length
      , ylabel = Time(s)
      , error bars/y dir = both
      , error bars/y explicit
      , error bars/error bar style = { line width=0.5pt }
      , error bars/error mark options =
          { rotate=90
          , red
          , mark size=4pt
          , line width=0.5pt
          }
      ]
      \addplot table
        [ x = N
        , y = Mean
        , y error = Stddev
        , col sep = comma
        ]
        {benchmark-optimized.csv};
    \end{axis}
    \end{tikzpicture}
    \caption{B subfigure}\label{fig:bench1:b}
  \end{minipage}
  \caption{Blah blah blah TODO}
  \label{fig:bench1}
\end{figure}

\chapter{Conclusions}
\label{cha:conclusions}

