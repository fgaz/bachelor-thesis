\chapter*{Abstract}
\label{abstract}

\addcontentsline{toc}{chapter}{Abstract}

When developing an Haskell program and using advanced type system features such as GADTs (Generalized Algebraic Data Types), it is often necessary to write proofs to make the program typecheck.
When they are in the form of a singleton (a type with a single inhabitant), such proofs can end up increasing the computational complexity of the program.
Since the proof can only return one possible value, one would expect that the compiler optimized it to a constant operation.
This is not the case though, due to the pervasive presence of another inhabitant: $\bot$, which most often manifests in the form of nontermination.
By employing a totality checker and a whitelist of inhabited types, we developed a GHC plugin that can perform the aforementioned optimization in a sound way.
Finally, we will demonstrate how it can bring the complexity of some proof-heavy expressions from $O(n^2)$ down to $O(n)$.

